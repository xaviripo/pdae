\documentclass{../pdae} %\documentclass[12pt]{../pdae}
% Necesario aquí para que lea bien el shorttitle
% Título en la cabecera

\shorttitle{Pràctica 3:\\Timers i Interrupcions}
\title{Creació de Funcions de Retard i un Rellotge en Temps Real mitjançant
l’ús de Timers i Interrupcions}

\author{
    Iván Canales\\
    Xavier Ripoll
}

\date{23 de març de 2018} % TODO poner la fecha de entrega

\begin{document}
\maketitle

\section{Introducció}

\subsection{Objectius}
% Què es vol fer a la pràctica.

La pràctica té dues parts principals.

Abans de fer cap de les dues, ens cal configurar els \textit{timers} del
microcontrolador per tal de poder utilitzar-los.

En el primer exercici, refactoritzem la part del
codi de la pràctica 2 en què es permetia modificar la freqüència i direcció
dels LEDs de la placa inferior. La idea és emprar els \textit{timers} del
microcontrolador per mesurar el temps amb més precisió i controlar millor
els casos excepcionals (e.g. quan al pujar o baixar la freqüència hi ha un
\textit{overflow} o \textit{underflow}).

La segona part consisteix en dissenyar tant la interfície (per la pantalla LCD)
com la implementació (novament amb \textit{clocks} i \textit{timers}) d'un
rellotge que porti l'hora i una alarma, ambdues programables.


\subsection{Recursos utilitzats}
% Quins recursos del Microcontrolador, Placa d’Experimentació i Robot es fan servir.
Utilitzarem el \texttt{ACLK} com a \textit{clock} base per al \textit{timer}
que controla els LEDs i el rellotge \texttt{SMCLK} per controlar l'hora i
l'alarma.
% TODO ^ poner los clocks que usamos para cada ejercicio

De la placa \textit{Boosterpack MK II} volem fer servir tant el
\textit{joystick} com els polsadors S1 i S2 com a entrades, a més a més de la
pantalla LCD com a sortida (en la que es mostrarà la freqüència dels LEDs,
l'hora i l'alarma).

De la placa d'experimentació, novament farem servir els vuit LEDs del port P7
per mostrar la llum que es va ``desplaçant''.


\subsection{Configuració dels recursos}
% Com s’han configurat els diferents recursos.

\subsection{Funcions utilitzades}
% Com i per quines funcions es fan servir.

\subsection{Problemes}
% Problemes que han sorgit (que no siguin de compilació) i com s’han solucionat.

\subsection{Conclusions}

\section{Comentari del codi}

\section{Diagrames de flux}

\end{document}
