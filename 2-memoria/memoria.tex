\documentclass[12pt,a4paper]{article}
\usepackage{graphicx}
\usepackage[utf8]{inputenc}
\usepackage{titling}
\usepackage{hyperref}
\usepackage{fancyhdr}
 

\rfoot{Page }

\title{Configuració de Ports}
\author{
    Iván Canales% Martín
    \\ 
    Xavier Ripoll% Echeveste 
}

\date{March 2018}

\pagestyle{fancy}
\fancyhf{}
\chead{Programació d'Arquitectures Encastades}
\rhead{\thepage}
\lhead{\theauthor}

\begin{document}

%%%%%%%%%%%%%%%%%%%%%%%%%%%%%%%%%%%%
%%%%%%%%%%%% TITLE PAGE %%%%%%%%%%%%
%%%%%%%%%%%%%%%%%%%%%%%%%%%%%%%%%%%%

% un dia de estos lo hago plantilla
\begin{titlepage}
	\centering
	{\scshape\LARGE Universitat de Barcelona \par}
	\vspace{2cm}
	{\scshape\Large Pràctica 2:\par} 
	\vspace{1cm}
	{\huge\bfseries \thetitle \par}

    \vfill
    \large\theauthor
	\vfill
	\raggedleft
    
    \par
    
    %\hrulefill\par
    {\scshape Programació d'Arquitectures Encastades\par}
    \texttt{}{Curs 17/18\par} %Universitat de Barcelona
    \today

% Bottom of the page
	
\end{titlepage} \pagebreak
%%%%%%%%%%%%%%%%%%%%%%%%%%%%%%%%%%%%

\section{Objectius}
% Què es vol fer a la pràctica. 

\section{Recursos utilitzats}
% Quins recursos del Microcontrolador, Placa d’Experimentació i Robot es fan servir.

\section{Configuració dels recursos}
% Com s’han configurat els diferents recursos.

\section{Funcions utilitzades}
% Com i per quines funcions es fan servir.

\section{Problemes}
% Problemes que han sorgit (que no siguin de compilació) i com s’han solucionat.

\section{Conclusions}


\end{document}